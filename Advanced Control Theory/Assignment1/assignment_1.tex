\documentclass[paper=a4, fontsize=11pt]{scrartcl} % A4 paper and 11pt font size

\usepackage[T1]{fontenc} % Use 8-bit encoding that has 256 glyphs
\usepackage{fourier} % Use the Adobe Utopia font for the document - comment this line to return to the LaTeX default
\usepackage[english]{babel} % English language/hyphenation
\usepackage{amsmath,amsfonts,amsthm} % Math packages
%\usepackage{sectsty} % Allows customizing section commands
\usepackage{blindtext}
\usepackage[utf8]{inputenc}
\usepackage{graphicx}
\graphicspath{{images/}}
%\allsectionsfont{\left\normalfont\scshape} % Make all sections centered, the default font and small caps
\usepackage{fancyhdr} % Custom headers and footers
\pagestyle{fancyplain} % Makes all pages in the document conform to the custom headers and footers
\fancyhead{} % No page header - if you want one, create it in the same way as the footers below
\fancyfoot[L]{} % Empty left footer
\fancyfoot[C]{\thepage} % put page number in center footer
\fancyfoot[C]{} % Empty right footer
\renewcommand{\headrulewidth}{0pt} % Remove header underlines
\renewcommand{\footrulewidth}{0pt} % Remove footer underlines
\setlength{\headheight}{} % Customize the height of the header

\numberwithin{equation}{section} % Number equations within sections (i.e. 1.1, 1.2, 2.1, 2.2 instead of 1, 2, 3, 4)
\numberwithin{figure}{section} % Number figures within sections (i.e. 1.1, 1.2, 2.1, 2.2 instead of 1, 2, 3, 4)
\numberwithin{table}{section} % Number tables within sections (i.e. 1.1, 1.2, 2.1, 2.2 instead of 1, 2, 3, 4)

\setlength\parindent{0pt} % Removes all indentation from paragraphs - comment this line for an assignment with lots of text

%----------------------------------------------------------------------------------------
%	TITLE SECTION
%----------------------------------------------------------------------------------------
\title{	
\normalfont \normalsize 
\textsc{EENG 517}\\ 
\huge In Class Problem 1 \\ % The assignment title
}

\author{Dana Martin} % Your name

\date{\normalsize Due: 01/21/15}

\begin{document}

\maketitle % Print the title

%----------------------------------------------------------------------------------------
%	PROBLEM 1
%----------------------------------------------------------------------------------------
\begin{section}{}
Derive a state-space controller form for
\begin{align}
\begin{split}
\ddot{y}+a_1\dot{y}+a_oy &= b_2\ddot{u}+b_1\dot{u}+b_ou
\end{split}
\end{align}
From examples and literature found, the first step is to take the Laplace transform of the above equation.
\begin{align}
\begin{split}
s^2Y(s)+a_1sY(s)+a_oY(s)=b_2s^2U(s)+b_1sU(s)+b_oU(s)\\
\end{split}
\end{align}
To calculate the transfer function for the above system, we solve for the ratio $\frac{Y(s)}{U(s)}$
\begin{align}
\begin{split}
\frac{Y(s)}{U(s)}&=\frac{b_2s^2+b_1s+b_o}{s^2+a_1s+a_o}
\end{split}
\end{align}
We now multiply equation 0.3 by $\frac{Z(s)}{Z(s)}$, then write expressions for Y(s) and U(s)
\begin{align}
\begin{split}
\frac{Y(s)}{U(s)}&=\frac{b_2s^2+b_1s+b_o}{s^2+a_1s+a_o}\Bigg(\frac{Z(s)}{Z(s)}\Bigg)\\
Y(s)&=b_2s^2Z(s)+b_1sZ(s)+b_oZ(s)\\
U(s)&=s^2Z(s)+a_1sZ(s)+a_oZ(s)
\end{split}
\end{align}
Now take the inverse Laplace Transform and we find:
\begin{align}
\begin{split}
y=b_2\ddot{z}+b_1\dot{z}+b_oz\\
u=\ddot{z}+a_1\dot{z}+a_o
\end{split}
\end{align}
Now we can define the state variables
\begin{align*}
x_1&=z	 &		\dot{x_1}&=\dot{z}=x_2\\
x_2&=\dot{z} &	\dot{x_2}&=\ddot{z}=u-a_1\dot{z}-a_oz\\
x_3&=\ddot{z}          
\end{align*}
We can now write the output as
\begin{align*}
y=b_2x_3+b_1x_2+b_ox_1\Longrightarrow                         & b_2\ddot{z}+b_1\dot{z}+b_oz\\
=& b_2(u-a_1\dot{z}-a_oz)+b_1\dot{z}+b_oz\\
=&\dot{z}(b_1-a_1)+z(b_o-a_o)+b_2u
\end{align*}

The state space-equations are:\\

\begin{align}
\boldsymbol{\dot{x}}&=\boldsymbol{Ax}+\boldsymbol{B}u\Rightarrow \begin{bmatrix} \dot{x_1}\\ \dot{x_2} \end{bmatrix} =\begin{bmatrix}
0 & 1\\-a_o & -a_1
\end{bmatrix}\begin{bmatrix}
x_1\\x_2
\end{bmatrix}+ \begin{bmatrix}0\\1\end{bmatrix}u\\
\boldsymbol{y}&=\boldsymbol{Cx}+\boldsymbol{D}u \Rightarrow \begin{bmatrix}
(b_o-a_o) & (b_1-a_1)\end{bmatrix} \begin{bmatrix}
x_1 \\ x_2
\end{bmatrix}+\begin{bmatrix}
b_2
\end{bmatrix}u
\end{align}
 
\end{section}

\end{document}